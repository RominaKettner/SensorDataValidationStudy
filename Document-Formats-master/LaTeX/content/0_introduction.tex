\section{Introduction}

Sensors promising to measure your physical activity, sleep quality and stress level can be found in many devices. ....  





The purpose of the study is to evaluate the consumer wearables (Apple Watch, MS Band II, Polar chest straps) as a suitable wearable to infer stress in a controlled lab environment. Therefore,we put participants in a cognitive demanding, stressful situation and collected sensing data from the consumer devices and professional sensors, namely the Nexus Kit. For comparison, we recorded a baseline consisting of a relaxation period. 

Conducting our study, we aimed to investing the following research questions:
\begin{enumerate}
	\item How do the collected physiological data differ between the different consumer good devices under limited physical activity?
	\item How do the collected physiological data differ between the different consumer good devices under high physical activity?
	\item How does physical data activity affect the data recording of physiological signals in general? \todo{das muessen wir moeglicherweise raus nehmen, weil ich nicht sicher bin inwiefern wir dazu Aussagen machen können}
	\item What are differences regarding the data collection between the different measures (EDA, ECG, Skin Temperature) with respect to the robustness?
	\item How does the user perception of emotions differ between limited and high physical activity under stressful conditions? \todo{hier wollte ich noch die SAM Ergebnisse einbauen, aber die Frage laesst sich vermutlich erst formulieren, wenn wir wissen ob/was rauskommt} 
\end{enumerate}

%Consequently, we contribute user study investigating (a) the amount of perceived stress , (b) the development of two wrist-worn stimulation giving prototypes, and (c) a preliminary as well as a user study exploring and comparing different tactile stimulation patterns.


 


