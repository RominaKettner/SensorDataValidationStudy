\section{Introduction and Related Work}


The number of notifications is continuously increasing. In particular, mobile devices such as smart watches or smartphones generate 65 notifications on average per day~\cite{Pielot:2014:ISM:2628363.2628364}. These notifications are communicated via visual, auditory, and tactile cues based on the user's current situation. For example, during meetings, users prefer tactile feedback while auditory notifications are desirable when the user is at home and placed his phone somewhere. Occasions in which tactile feedback is used, are mainly characterized by the fact that the user is highly engaged in other tasks (e.g., meetings, presentations) and, thus, might be easily distracted and stressed by incoming notifications. 

Besides vibrational feedback as known from consumer devices, research proposed different tactile feedback methods that can be integrated into mobile and wearable devices.
Examples include simple tactile stimulation via tapping, dragging, squeezing, and twisting~\cite{nakagaki_LineFORM,6226397}. 
Additionally, pressure-based feedback yields advantages such as unobtrusiveness~ \cite{pohl2015wrist}. 
There are also other approaches using Electrical Muscle Stimulation~\cite{Schneegass:2016:ENI:2957265.2962663} or changes in temperature~\cite{song_Hot}.
These methods are designed to gain the user's attention as fast as possible. As a performance measure, research therefore investigates the time needed for perceiving the feedback cue.
In contrast, we investigate how we can generate a tactile stimulation pattern which induces less stress to the user. 
We thereby focus on pressure-based stimulation as well as vibrotactile feedback as state of the art. Alvina et al. explored spatiotemporal vibrotactile patterns on different body parts and confirmed it's recognizability \cite{alvina2015omnivib}. 
In our work, we investigate how tactile feedback can be designed to suit stressful situations deriving the feedback pattern from physiological signals.

Consequently, we contribute (a) a concept for deriving tactile stimulation patterns from physiological signals, (b) the development of two wrist-worn stimulation giving prototypes, and (c) a preliminary as well as a user study exploring and comparing different tactile stimulation patterns.


 


