\section{User study: Comparing varied consumer goods under low and high physical activity in differently cognitive demanding situations}

In our user study, we compared the validity of three consumer goods () in a within-subject design.  

\subsection{Measures} 

- Physiological data: EDA, ECG, Skin temperature
- Arousal and its subscales accessed via SAM
- single-item to self-rate the stress level 
- other sensors e.g. accelerometer

\subsection{Conditions and Tasks}

- explaining the two overall conditions: low physical activity, namely sitting while listening to music and high physical activity which means running 
- describing the two different tasks including the duration and the stimulus material \todo{here we can also refer to and cite the papers we took the study design and the tasks from}
- referring to a study design figure that depicts the sequence of events

\subsection{Participants and Procedure}

For this laboratory study, we recruited xx participants via university mailing lists and personal acquisition. The mean age was xx ($ SD = $) among the xx females and xx males. 
BLBLAB